\documentclass[a4paper,10pt]{article}
\usepackage[vcentering,dvips]{geometry}

\newcommand{\linux}{\mbox{\sc LINUX}}
\newcommand{\awk}{\mbox{\sc awk}}
\newcommand{\sort}{\mbox{\sc sort}}

\newcommand {\articleTitle}{Informe Trabajo Pr\'actico 1}

% misc
\newcommand {\murl}[2]{\href{#1}{#2}}

% marcas / nombres de cosas usadas 
\usepackage[spanish]{babel}
\usepackage{graphicx}
\usepackage{graphics}
\usepackage[utf8]{inputenc}
\usepackage{subfigure}
\usepackage[pdftitle=\articleTitle
	pdfauthor={},
	pdfsubject={},
	pdfkeywords={} 
]{hyperref}

\usepackage{verbatim}
\usepackage{rotating}
% inclusion de codigo fuentes
\usepackage{listings}
\lstloadlanguages{c}

\author
{
	Pablo Giorgi,
	Santiago Perez De Rosso,
	Luciano Zemin
}

\date{Abril 2011}
% Title
\title{\articleTitle}

\begin{document}
\bibliographystyle{acm}
\maketitle

\tableofcontents

\section{Introducción}

En el presente informe se presenta las soluciones correspondientes al descifrado
de los archivos provistos por la c\'atedra y el an\'alisis de las cuestiones
planteadas en el punto 4 del Trabajo Pr\'actico 1.

\section{Descifrado de los archivos provistos por la c\'atedra}

Se desencriptaron los 6 archivos provistos por la c\'atedra
\emph{fun6AESCBC.wav}, \emph{fun6AESCFB.wav}, \emph{fun6AESOFB.wav},
\emph{fun6DESCBC.wav}, \emph{fun6DESOFB.wav} utilizando la clave \emph{sorpresa}
con el algoritmo y modo indicado en el nombre del archivo. Se obtuvo
el mismo fragmento de la canción \emph{El matador} de los \emph{Fabulosos Cadillacs}
en todos los casos.

\section{An\'alisis de las cuestiones planteadas en el punto 4 del Trabajo Pr\'actico 1}

Para analizar el primer punto se toma el archivo \emph{fa-do.wav}

\end{document}
